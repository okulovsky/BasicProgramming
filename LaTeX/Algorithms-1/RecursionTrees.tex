\documentclass[24pt,pdf,hyperref={unicode},aspectratio=169]{beamer}
\usepackage[utf8]{inputenc}
\usepackage[russian]{babel}
\usepackage{tikz}
\begin{document}

\begin{frame}[fragile]\frametitle{Дерево рекурсивных вызовов}
\begin{columns}[t]
\column{0.4\textwidth}

\begin{verbatim}
static void Make(int n)
{
    for (int i=n-1; i>=0; i--)
    {
        Console.Write(i + " ");
        Make(i);
    }
}
\end{verbatim}

\column{0.5\textwidth}

\uncover<+->{}
\begin{tikzpicture}[y=-0.8cm]

\uncover<+->{
\node[draw] (Make2) at (0,0) {\tt Make(2)};
}

\uncover<+->{
\node[draw] (Make21) at (-1,1) {\tt Make(1)};
\draw (Make2) edge[->] (Make21);
}

\uncover<+->{
\node[draw] (Make210) at (-1,2) {\tt Make(0)};
\draw (Make21) edge[->] (Make210);
}

\uncover<+->{
\node[draw] (Make20) at (1,1) {\tt Make(0)};
\draw (Make2) edge[->] (Make20);
}

\uncover<+->{
\node[draw] (Make3) at (1,-1) {\tt Make(3)};
\draw (Make3) edge[->] (Make2);
}

\uncover<+->{
\node[draw] (Make31) at (3,1) {\tt Make(1)};
\draw (Make3) edge[->] (Make31);
}

\uncover<+->{
\node[draw] (Make310) at (3,2) {\tt Make(0)};
\draw (Make31) edge[->] (Make310);
}

\uncover<+->{
\node[draw] (Make30) at (5,1) {\tt Make(0)};
\draw (Make3) edge[->] (Make30);
}
\end{tikzpicture}

\uncover<+->{$$
p(0)=1
$$}

\uncover<+->{$$
p(n)=1+\sum_{i=0}^{n-1} p(i)
$$}

\end{columns}
\end{frame}

\begin{frame}\frametitle{Расчет количества вызовов}
\uncover<+->{$$
p(0)=1
$$}
\uncover<+->{$$
p(n)=1+\sum_{i=0}^{n-1} p(i)
$$}
\uncover<+->{$$
p(0)=1,\ p(1)=1+1=2,\ p(2)=1+2+1=4,\ p(3)=1+4+2+1=8,\ p(4)=16
$$}
\uncover<+->{$$
p(n)=2^n?
$$}
\uncover<+->{$$
1+\sum_{i=0}^{n-1} 2^i=1+2^n-1=2^n
$$}

\end{frame}

\begin{frame}[fragile]\frametitle{Дерево рекурсивных вызовов}
\begin{columns}[t]
\column{0.4\textwidth}

\begin{verbatim}
static int Make(int x, int y)
{
	if (x<=0 || y<=0) return 1;
	return Make(x-1,y)+Make(x,y-1);
}
\end{verbatim}

\column{0.5\textwidth}

\newcommand{\nn}[3]{\node[draw] (M#3) at (#2) {(#1)};}
\newcommand{\nnc}[4]{
\uncover<+->{
\nn{#1}{#2}{#3} 

\draw (M#4) edge[->] (M#3);
}}

\uncover<+->{}
\begin{tikzpicture}[x=1.3cm]
\uncover<+->{
\nn{2,2}{0,0}{22}
}
\nnc{1,2}{0,1}{12}{22}
\nnc{0,2}{0,2}{02}{12}
\nnc{1,1}{1,1}{11}{12}
\nnc{0,1}{1,2}{01}{11}
\nnc{1,0}{2,1}{10}{11}
\nnc{2,1}{3,0}{21}{22}
\nnc{1,1}{3,1}{11A}{21}
\nnc{0,1}{3,2}{01A}{11A}
\nnc{1,0}{4,1}{10A}{11A}
\nnc{2,0}{4,0}{20B}{21}
\end{tikzpicture}
\end{columns}
\end{frame}

\begin{frame}\frametitle{Расчет количества итераций}
$$
p(0,y)=1
$$
$$
p(x,0)=1
$$
\uncover<+->{$$
p(x,y)=1+p(x-1,y)+p(x,y-1)
$$}
\only<+>{$$
\begin{array}{c|c c c c c}
  & 0 & 1 & 2 & 3 & 4 \\
\hline
0 & 1 & 1 & 1 & 1 & 1 \\
1 & 1 & 3 & 5 & 7 & 9 \\
2 & 1 & 5 & 11 & 18 & 28 \\
3 & 1 & 7 & 18 & 37 & 66 \\
4 & 1 & 9 & 28 & 66 & 133 
\end{array}
$$}
\uncover<+->{$$
p(x,y)\le 3^{x+y}
$$}
\uncover<+->{$$
p(0,0)=1\le 3^0
$$}
\uncover<+->{$$
p(x,y)=1+p(x-1,y)+p(x,y-1)\le 3^{x+y-1}+3^{x+y-1}+3^{x+y-1} = 3^{x+y}
$$}


\end{frame}

\begin{frame}\frametitle{Расчет количества итераций}
$$
p'(0,y)=1
$$
$$
p'(x,0)=1
$$
$$
p'(x,y)=p'(x-1,y)+p'(x,y-1)<p(x,y)
$$
\uncover<+->{$$
\begin{array}{c|c c c c c}
  & 0 & 1 & 2 & 3 & 4 \\
\hline
0 & 1 & 1 & 1 & 1 & 1 \\
1 & 1 & 2 & 3 & 4 & 5 \\
2 & 1 & 3 & 6 & 10 & 15 \\
3 & 1 & 4 & 10 & 20 & 35 \\
4 & 1 & 5 & 15 & 35 & 70 
\end{array}
$$}
\uncover<+->{
$$
p'(x,y)=C^{x}_{x+y}=\frac{(x+y)!}{x!y!}
$$}
\end{frame}

\end{document}
