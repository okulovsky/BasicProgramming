\documentclass[24pt,pdf,hyperref={unicode},aspectratio=169]{beamer}
\usepackage[utf8]{inputenc}
\usepackage[russian]{babel}
\begin{document}

\begin{frame}\frametitle{Линейный поиск}
{\large

$$
\begin{array}{c c c c c c c c c}
\only<2>{\downarrow} & 
\only<3>{\downarrow} & 
\only<4>{\downarrow} & 
\only<5>{\downarrow} & 
\only<6>{\downarrow} & 
\only<7>{\downarrow} & 
\only<8>{\downarrow} & 
\only<9>{\downarrow} & 
\only<10>{\downarrow} \\
5 & 2 & 3 & 13 & 8 & 1 & 12 & 10 & 15 \\
\end{array}
$$}
\end{frame}

\begin{frame}[fragile]\frametitle{Алгоритм линейного поиска}

\begin{verbatim}
int Find(int[] array, int query)
{
    for (int i=0;i<array.Length;i++)
        if (array[i]==query)
            return i;
    return -1;
}
\end{verbatim}

\end{frame}

\begin{frame}\frametitle{Бинарный поиск}
{\large
Поиск $7$

$$ 
\begin{array}{c c c c c c c c c c c c c}
&&&& \only<9>{\downarrow} \\
\only<2-5>{[}
&&&
\only<5>{\downarrow}
&
\only<6-10>{[}\only<10>{]}
&
\only<7>{\downarrow}\only<8-9>{]}
&
\only<3>{\downarrow}\only<4-7>{]}
&&&&&&
\only<2-3>{]}
\\
1 & 3 & 4 & 6 & 7 & 7 & 9 & 10& 12 & 12 & 12 & 15 & 20 \\ \hline
0 & 1 & 2 & 3 & 4 & 5 & 6 & 7 & 8 & 9 & 10 & 11 & 12 \\
\end{array}
$$}
\end{frame}

\begin{frame}\frametitle{Бинарный поиск}
Поиск $4$

$$
\begin{array}{c c c c c}
\only<2-5>{[} &
\only<5>{\downarrow}&
\only<6>{[}\only<3>{\downarrow}\only<4->{]}&
&
\only<2-3>{]}
\\
1 & 3 & 5 & 7 & 9 \\
\hline
0 & 1 & 2 & 3 & 4 
\end{array}
$$
\end{frame}



\begin{frame}[fragile]\frametitle{Алгоритм бинарного поиска}
\begin{columns}[t]
\column{0.5\textwidth}
{\small
\begin{verbatim}
int Find(int[] array, int query)
{
    int left=0;
    int right=array.Length-1;
    while(left<right)
    {
        var middle=(left+right)/2;
        if (query<=array[middle])
            right=middle;
        else
            left=middle+1;
    }
    if (array[right]==query) 
        return right;
    return -1; 
}
\end{verbatim}
}
\column{0.5\textwidth}

\only<2-3>{Утверждение. На каждой итерации значение {\tt right-left} уменьшается.\\[0.5cm]}
\only<3>{Следовательно, рано или поздно оно станет неположительным, т.е. условие {\tt left<right} нарушится.}
\only<4-5>{Утверждение. Если в массиве есть искомые значения, то на каждой итерации хотя бы одно из них находится между индексов {\tt left} и {\tt right}\\ [0.5cm]}
\only<5>{Следовательно, если по завершению массива значение не обнаружено, то его не было в исходном массиве.}
\only<6-8>{Утверждение. На каждой итерации значение {\tt right-left} уменьшается не менее чем вдвое.\\[0.5cm]}
\only<7-8>{Утверждение. Если последовательно уменьшать число $N$ путем целочисленного деления на $2$, то в ноль оно обратится не позже, чем через $\lfloor \log_2 N\rfloor+1$ шагов\\[0.5cm]}
\only<8>{Следовательно, алгоритм совершит порядка $\log {\tt array.Length}$ шагов, и имеет оценку сложности $\Theta(\log_n)$.}

\end{columns}
\end{frame}

\begin{frame}

$$
\frac{l+r}{2}=l+\frac{r-l}{2}
$$

Вариант 1.

$$
r'=l+\frac{r-l}{2},\ l'=l
$$

$$
r'-l'=l+\frac{r-l}{2}-l=\frac{r-l}{2}<r-l
$$

Вариант 2.

$$
r'=r,\ l'=1+l+\frac{r-l}{2}
$$

$$
r'-l'=r-l-\frac{r-l}{2}-1<\frac{r-l}{2}<r-l
$$

\end{frame}

\begin{frame}

$$
l=\lceil \log_2 N \rceil +1
$$

$$
2^l>N
$$

$$
1=\frac{2^l}{2^l}>\frac{N}{2^l}
$$

\end{frame}

\begin{frame}\frametitle{Анализ алгоритма}

\begin{enumerate}
\item Доказательство корректности:
\begin{enumerate}
\item Алгоритм всегда останавливается?
\item Алгоритм всегда возвращает правильный ответ?
\end{enumerate}
\item Оценка сложности алгоритма
\end{enumerate}
\end{frame}



\end{document}
