\documentclass[24pt,pdf,hyperref={unicode},aspectratio=169]{beamer}
\usepackage[utf8]{inputenc}
\usepackage[russian]{babel}
\begin{document}

\begin{frame}\frametitle{Задача}
\uncover<+->{{\bf Задача} -- это соответствие, определяющее зависимость выхода (слова) от входа (слова).}\\[20pt]

\uncover<+->{$\Sigma$ -- алфавит (произвольное конечное множество, элементы которого интерпретируются как символы)}\\[20pt]

\uncover<+->{$\Sigma^*$ -- множество всех слов из букв алфавита $\Sigma$.}\\[20pt]

\uncover<+->{$\rho\subset\Sigma^*\times\Sigma^*$ -- бинарное отношение, определяющее задачу. Пара $(x,y)\in\rho$ показывает, что $y$ является допустимым выходом для входа $x$}
\end{frame}

\begin{frame}\frametitle{Алгоритм и программа}

\uncover<+->{{\bf Алгоритм} -- это последовательность элементарных операций, обрабатывающая входную строку $x$ для получения выходной строки $y$ такой, что $(x,y)\in\rho$}\\[20pt]

\uncover<+->{Под элементарной операцией в этом курсе мы будем понимать операции, исполняющиеся непосредственно на процессоре: сложение чисел, умножение и т.д.}\\[20pt]

\uncover<+->{{\bf Программа} -- это алгоритм, выраженный на некотором языке, который может быть транслирован в элементарные операции}

\end{frame}

\begin{frame}\frametitle{Сложность алгоритма}

\uncover<+->{{\bf Временная сложность алгоритма} -- это функция $f(n)$, $f:\mathbb{N}\rightarrow\mathbb{N}$, показывающая точную верхнюю границу количества элементарных операций, необходимых для завершения работы алгоритма, в зависимости от количества символов во входе}\\[1cm]
\uncover<+->{{\bf Емкостная сложность алгоритма} -- аналогичная оценка для \textit{дополнительной} памяти, необходимой для анализа входа. Память, использующаяся для хранения входа, не учитывается.}
\end{frame}


\begin{frame}\frametitle{Сложность алгоритма}
\begin{columns}
\column{0.6\textwidth}
{\tt
var n=Console.ReadLine().Length;

\ 

var sum=0;

for (int i=0;i<n;i++)

\ \ \ for(int j=0;j<2*i;j++)

\ \ \ \ \ \ sum++;

\ 

Console.WriteLine(sum);
}

\column{0.4\textwidth}
\uncover<+->{}

$$
\uncover<+->{0+2+4+\ldots+2(n-1)}\uncover<+->{=n(n-1)}
$$ 

$$
\uncover<+->{f(n)=n(n-1)(2_{++}+1_{*}+1_{<})+}
$$ $$
\uncover<+->{+n(1_=+1_{++}+1_<)+2_=+RL+WL}
$$ $$
\uncover<+->{=k_{W}n+k_{R}\log_{10}n(n-1)+4n^2-n+2}
$$


\end{columns}
\end{frame}


\begin{frame}\frametitle{О-символика}
$$
\begin{array}{l l l l}
\uncover<1->{f(n)=o(g(n))}
	& \uncover<2->{
		\begin{array}{l}
			\forall k>0\ \exists n_0\ \forall n>n_0\\ 
			f(n)<k\cdot g(n) 
	  \end{array}}
	& \uncover<7->{\Leftrightarrow \lim_{n\rightarrow \infty} \frac{f(n)}{g(n)}=0}
	& \uncover<10->{f(n)\prec g(n)}\\
\uncover<3->{f(n)=O(g(n))} 
	& \uncover<4->{
		\begin{array}{l}
		\exists k>0\ \exists n_0\ \forall n>n_0 \\
		 f(n)<k\cdot g(n)
		\end{array}} 
	& \uncover<8->{\Leftrightarrow \lim_{n\rightarrow \infty} \frac{f(n)}{g(n)}<\infty}
	& \uncover<11->{f(n)\preceq g(n)}\\
\uncover<5->{f(n)=\Theta(g(n))}
	& \uncover<6->{
		\begin{array}{l}
		\exists k_1,k_2>0\ \exists n_{0}\ \forall n>n_{0}  \\
		k_1\cdot g(n)<f(n)<k_2 \cdot g(n) 
		\end{array}}
	& \uncover<9->{\Leftarrow \lim_{n\rightarrow \infty} \frac{f(n)}{g(n)}=c>0}
	& \uncover<12->{f(n) \approx g(n)} \\
\end{array}
$$
\end{frame}

\begin{frame}
$$ f(n)=n(3+\sin n) $$

$$ g(n)=n $$

$$ g(n) < f(n) < 5g(n) $$

$$ \lim_{n\rightarrow\infty}\frac{f(n)}{g(n)}=\lim_{n\rightarrow\infty} (3+\sin n) $$


\end{frame}

\begin{frame}\frametitle{Оценка сложности}
\uncover<+->{$$
f(n)=4n^2+kn+2
$$}
\uncover<+->{$$
g(n)=n^2
$$}
\uncover<+->{$$
\lim_{n\rightarrow\infty}\frac{4n^2+kn+2}{n^2}=
$$}
\uncover<+->{$$
\lim_{n\rightarrow\infty}\frac{4n^2}{n^2}+\lim_{n\rightarrow\infty}\frac{kn}{n^2}+\lim_{n\rightarrow\infty}\frac{2}{n^2}=\uncover<+->{4}
$$}
\uncover<+->{$$
f(n)=\Theta(g(n))=\Theta\left(n^2\right)
$$}
\end{frame}



\begin{frame}\frametitle{Сложность алгоритма}
\begin{columns}
\column{0.6\textwidth}
{\tt
var n=int.Parse(Console.ReadLine());

\ 

var root=(int)Math.Sqrt(n);

for (int i=2;i<root;i++)

\ \ \ if (n \% i == 0) 

\ \ \ \{

\ \ \ \ \ \ Console.WriteLine("yes");

\ \ \ \ \ \ return;

\ \ \ \}

\ 

Console.WriteLine("no");
}

\column{0.4\textwidth}
\uncover<+->{}

$$
\uncover<+->{f(n)=\Theta\left(\sqrt{n}\right)}
$$

$$
n=\Theta\left(10^{|x|}\right)
$$

$$
\uncover<+->{f(|x|)=\Theta\left(\sqrt{10^{|x|}}\right)}
$$

\end{columns}
\end{frame}

\begin{frame}

Алгоритм со сложностью $f(n)$ называется:

\begin{itemize}

\item Если $f=\Theta\left(\log^k n\right)$: логарифмическим при $k=1$, полилогарифмическим при $k>1$.

\item Если $f=\Theta(n)$: линейным

\item Если $f=\Theta\left(n\log^k n\right)$: linearithmic при $k=1$, квазилинейным при $k>1$

\item Если $f=\Theta\left(n^k\right)$: полиномиальным, при $k=2$ квадратичным.

\item Если $f=\Theta\left(2^{n^k}\right)$: экспоненциальным.
\end{itemize}

\end{frame}

\end{document}
