\documentclass[24pt,pdf,hyperref={unicode},aspectratio=169]{beamer}
\usepackage[utf8]{inputenc}
\usepackage[russian]{babel}
\usepackage{tikz}

\usetikzlibrary{calc}
\usetikzlibrary{shapes}

\tikzstyle{dedge} = [draw,thick,->]
\tikzstyle{edge} = [draw,thick,-]
\tikzstyle{gedge} = [draw=green,thick,-]
\tikzstyle{redge} = [draw=red,thick,-]
\tikzstyle{ver} = [circle, draw=black]
\tikzstyle{verg} = [circle, draw=black, fill=gray]
\tikzstyle{verb} = [circle, draw=black, fill=black, text=white]

\deftranslation[to=russian]{Lemma}{Лемма}
\deftranslation[to=russian]{Theorem}{Теорема}


\begin{document}

\begin{frame}
\begin{center}
\begin{tikzpicture}[x=2cm,y=2cm]
\node[ver] (n0) at (0,0) {};
\node[ver] (n1) at (2,0) {};
\node[ver] (n2) at (1,1) {};
\node[ver] (n3) at (1,-1) {};
\path [edge] (n0) -- node[above] {2} (n2) -- node[above] {2} (n1);
\path [edge] (n0) -- node[below] {5} (n3);
\path [edge] (n2) -- node[left] {-4} (n3);
\end{tikzpicture}
\end{center}
\end{frame}

\begin{frame}
\begin{center}
\begin{tikzpicture}[x=2cm,y=-2cm]
\node[ver] (n0) at (0,0) {};
\node[ver] (n1) at (1,0) {};
\node[ver] (n2) at (2,0) {};
\node[ver] (n3) at (2,1) {};
\node[ver] (n4) at (1,1) {};
\node[ver] (n5) at (3,0) {};

\path[edge] (n0) -- node[above] {2} (n1) -- node[above] {2} (n2) -- node[right] {-3} (n3) -- node[below] {1} (n4) -- node[left] {-2} (n1);

\path[edge] (n2) -- node[above] {2} (n5);

\end{tikzpicture}
\end{center}
\end{frame}

\begin{frame}
\begin{lemma}
Если в графе нет циклов отрицательной длины, то кратчайший путь между любыми двумя вершинами не длиннее количества вершин в графе. 
\end{lemma}
\begin{proof}
В более длинном пути некоторая вершина встречается дважды, следовательно, есть цикл. По условию, его длина положительна, то есть удаление этого цикла из пути уменьшит длину пути.
\end{proof}
\end{frame}

\begin{frame}
\begin{columns}
\column{0.4\textwidth}
\begin{center}
\begin{tikzpicture}[x=2cm,y=2cm]
\node[ver] (n0) at (0,0) {$v_1$};
\node[ver] (n1) at (2,0) {$v_4$};
\node[ver] (n2) at (1,1) {$v_2$};
\node[ver] (n3) at (1,-1) {$v_3$};
\path [dedge] (n0) -- node[above] {2} (n2);
\path [dedge] (n2) -- node[above] {2} (n1);
\path [dedge] (n0) -- node[below] {5} (n3);
\path [dedge] (n3) -- node[left] {-4} (n2);
\end{tikzpicture}
\end{center}
\column{0.6\textwidth}
\begin{center}
\begin{tikzpicture}[y=-1cm]

\foreach \x in {1,2,3,4}
	\node at (\x,-1) {$v_\x$};

\foreach \y in {0,1,2,3,4}
	\node at (0,\y) {\y};

\uncover<+->{}
	
\uncover<+->{
\node (v10) at (1,0) {0};
}

\foreach \y/\x/\s/\py/\px in {
1/1/0/0/1,
1/2/2/0/1,
1/3/5/0/1,
2/1/0/1/1,
2/2/1/1/3,
2/3/5/1/3,
2/4/4/1/2,
3/1/0/2/1,
3/2/1/2/2,
3/3/5/2/3,
3/4/3/2/2,
4/1/0/3/1,
4/2/1/3/2,
4/3/5/3/3,
4/4/3/3/4,
}
{
\uncover<+->{
	\node (v\x\y) at (\x,\y) {\s};
	\path[dedge] (v\px\py) -- (v\x\y);
	}	
}
\end{tikzpicture}
\end{center}
\end{columns}
\end{frame}


\end{document}