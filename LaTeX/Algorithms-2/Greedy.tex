\documentclass[24pt,pdf,hyperref={unicode},aspectratio=169]{beamer}
\usepackage[utf8]{inputenc}
\usepackage[russian]{babel}
\usepackage{tikz}

\usetikzlibrary{calc}

\tikzstyle{dedge} = [draw,thick,->]
\tikzstyle{edge} = [draw,thick,-]
\tikzstyle{ver} = [circle, draw=black]
\tikzstyle{verg} = [circle, draw=black, fill=gray]
\tikzstyle{verb} = [circle, draw=black, fill=black, text=white]

\deftranslation[to=russian]{Lemma}{Лемма}


\newcommand{\seg}[4]{
\draw[draw=#4] (#1,#3) -- (#2,#3);
\draw[draw=#4] ($(#1,#3)-(0,0.1)$) -- ($(#1,#3)+(0,0.1)$);
\draw[draw=#4] ($(#2,#3)-(0,0.1)$) -- ($(#2,#3)+(0,0.1)$);
}
\newcommand{\segb}[3]{\seg{#1}{#2}{#3}{black}}

\begin{document}

\section{Комбинаторные задачи}

\section{Жадный алгоритм}

\section{Составление расписаний}

\begin{frame}
\begin{center}
\begin{tikzpicture}
\segb{1}{4}{0}
\segb{5}{8}{0}
\segb{9}{12}{0}

\segb{3}{6}{-1}
\segb{7}{10}{-1}

\end{tikzpicture}
\end{center}
\end{frame}

\begin{frame}
\begin{center}
\begin{tikzpicture}
\segb{1}{11}{0}
\segb{2}{4}{-1}
\segb{5}{7}{-1}
\segb{8}{10}{-1}
\end{tikzpicture}
\end{center}
\end{frame}

\begin{frame}
\begin{center}
\begin{tikzpicture}
\segb{1}{5}{0}
\segb{6}{10}{0}
\segb{4}{7}{-1}

\end{tikzpicture}
\end{center}
\end{frame}

\begin{frame}
\begin{center}
\begin{tikzpicture}[x=0.7cm]
\segb{1}{4}{0}
\segb{5}{8}{0}
\segb{9}{12}{0}
\segb{13}{16}{0}

\segb{7}{10}{-1}

\segb{3}{6}{-1}
\segb{3}{6}{-2}
\segb{3}{6}{-3}

\segb{11}{14}{-1}
\segb{11}{14}{-2}
\segb{11}{14}{-3}

\end{tikzpicture}
\end{center}
\end{frame}

\begin{frame}
\uncover<+->{
Доказательство корректности:

Пусть $s(x_i)$ -- начало промежутка $x_i$, $e(x_i)$ -- окончание.

Пусть $y_1,\ldots,y_n$ -- решение, найденное жадным алгоритмом, и $z_1,\ldots,z_m$ -- оптимальное решение.}

\uncover<+->{\begin{lemma}
Для любого $k$, $e(y_k)\le e(z_k)$.
\end{lemma}}

\uncover<+->{\begin{proof}
База индукции: жадный алгоритм выбирает $y_1$ так, что $e(y_1)$ -- минимально.

Шаг индукции: поскольку $e(y_{k-1})\le e(z_{k-1})$, то $z_k$ является допустимым для продолжения $y_1,\ldots,y_{k-1}$. $y_k$ -- элемент с минимальным $e$ из всех допустимых, следовательно, $e(y_k)\le e(z_k)$.
\end{proof}}

\uncover<+->{\begin{lemma}
$n\ge m$.
\end{lemma}}
\uncover<+->{\begin{proof}
Пусть $m>n$. Поскольку $e(y_n)<e(z_n)$, то $z_{n}$ допустим для $y_1,\ldots,y_n$. Но тогда жадный алгоритм включил бы его в эту последовательность.
\end{proof}
}
\end{frame}

\begin{frame}
\begin{center}
\begin{tikzpicture}[x=3cm, y=3cm]

\node[ver] (n0) at (0,0) {};
\node[ver] (n1) at (2,0) {};
\node[ver] (n2) at (1,1) {};
\node[ver] (n3) at (1,-1) {};

\path [edge] (n0) -- node[above] {5} (n1);
\path [edge] (n0) -- node[above] {1} (n2) -- node[above] {4} (n1);
\path [edge] (n0) -- node[below] {2} (n3) -- node[below] {2} (n1);

\end{tikzpicture}
\end{center}
\end{frame}

\begin{frame}
\begin{verbatim}
\end{verbatim}
\end{frame}

\end{document}