\documentclass[24pt,pdf,hyperref={unicode},aspectratio=169]{beamer}
\usepackage[utf8]{inputenc}
\usepackage[russian]{babel}
\begin{document}

\begin{frame}\frametitle{Задача}
\uncover<+->{{\bf Задача} -- это соответствие, определяющее зависимость выхода (слова) от входа (слова).}\\[20pt]

\uncover<+->{$\Sigma$ -- алфавит (произвольное конечное множество, элементы которого интерпретируются как символы)}\\[20pt]

\uncover<+->{$\Sigma^*$ -- множество всех слов из букв алфавита $\Sigma$.}\\[20pt]

\uncover<+->{$\rho\subset\Sigma^*\times\Sigma^*$ -- бинарное отношение, определяющее задачу.}\\[20pt]

\end{frame}

\begin{frame}\frametitle{Алгоритм и программа}

\uncover<+->{{\bf Алгоритм} -- это последовательность элементарных операций, обрабатывающая входную строку $x$ для получения выходной строки $y$ такой, что $(x,y)\in\rho$}\\[20pt]

\uncover<+->{Под элементарной операцией в этом курсе мы будем понимать операции, исполняющиеся непосредственно на процессоре: сложение чисел, умножение и т.д.}

\uncover<+->{{\bf Программа} -- это алгоритм, выраженный на некотором языке, который может быть транслирован в элементарные операции}

\end{frame}

\begin{frame}\frametitle{Сложность алгоритма}

\uncover<+->{{\bf Сложность алгоритма} -- это функция $f(n)$, $f:\mathbb{N}\rightarrow\mathbb{N}$, показывающая точную верхнюю границу количества элементарных операций, необходимых для завершения работы алгоритма, в зависимости от количества символов во входе}

\end{frame}

\begin{frame}\frametitle{Сложность алгоритма}
\begin{columns}
\column{0.6\textwidth}
{\tt
var n=Console.ReadLine().Length;

var sum=0;

for (int i=0;i<n;i++)

\ \ \ for(int j=0;j<2*i;j++)

\ \ \ \ \ \ sum++;

Console.WriteLine(sum);
}

\column{0.4\textwidth}
\uncover<+->{}

$$
\uncover<+->{f(n)}\uncover<+->{=0+2+4+\ldots+2n}\uncover<+->{=2n^2}
$$

\end{columns}
\end{frame}

\begin{frame}\frametitle{Сложность алгоритма}
\begin{columns}
\column{0.6\textwidth}
{\tt
var n=int.Parse(Console.ReadLine());

var sum=0;

for (int i=0;i<n;i++)

\ \ \ for(int j=0;j<2*i;j++)

\ \ \ \ \ \ sum++;
}

Console.WriteLine(sum);

\column{0.4\textwidth}
\uncover<+->{}

$$
\uncover<+->{f(n)=0+2+4+\ldots+2n=2n^2}
$$
$$
\uncover<+->{|x|=\lceil log_{10}(n) \rceil}
$$
$$
\uncover<+->{10^{|x|-1}\le n \le 10^{|x|}}
$$
$$
\uncover<+->{
2\left(10^{|x|-1}\right)^2 \le f(|x|) \le 2\left(10^{|x|}\right)^2}
$$
$$
\uncover<+->{
f(|x|) = 2\left(10^{|x|}\right)^2}
$$


\end{columns}
\end{frame}

\begin{frame}\frametitle{Сложность алгоритма}
\begin{columns}
\column{0.6\textwidth}
{\tt
var n=int.Parse(Console.ReadLine());

var root=(int)Math.Sqrt(n);

\ 

for (int i=2;i<root;i++)

\ \ \ if (n \% i == 0) 

\ \ \ \{

\ \ \ \ \ \ Console.WriteLine("yes");

\ \ \ \ \ \ return;

\ \ \ \}

\ 

Console.WriteLine("no");
}

\column{0.4\textwidth}
\uncover<+->{}

$$
\uncover<+->{f(n)=\sqrt{n}}
$$

$$
\uncover<+->{f(|x|)=\sqrt{10^{|x|}}}
$$

\end{columns}
\end{frame}

\begin{frame}\frametitle{О-символика}
\end{frame}

\begin{frame}\frametitle{О-символика}
\end{frame}

\begin{frame}\frametitle{Классы вычислительной сложности}

\end{frame}

\end{document}
